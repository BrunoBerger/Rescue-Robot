%% bare_conf_compsoc.tex
%% V1.4b
%% 2015/08/26
%% by Michael Shell
%% See:
%% http://www.michaelshell.org/
%% for current contact information.
%%
%% This is a skeleton file demonstrating the use of IEEEtran.cls
%% (requires IEEEtran.cls version 1.8b or later) with an IEEE Computer
%% Society conference paper.
%%
%% Support sites:
%% http://www.michaelshell.org/tex/ieeetran/
%% http://www.ctan.org/pkg/ieeetran
%% and
%% http://www.ieee.org/

%%*************************************************************************
%% Legal Notice:
%% This code is offered as-is without any warranty either expressed or
%% implied; without even the implied warranty of MERCHANTABILITY or
%% FITNESS FOR A PARTICULAR PURPOSE!
%% User assumes all risk.
%% In no event shall the IEEE or any contributor to this code be liable for
%% any damages or losses, including, but not limited to, incidental,
%% consequential, or any other damages, resulting from the use or misuse
%% of any information contained here.
%%
%% All comments are the opinions of their respective authors and are not
%% necessarily endorsed by the IEEE.
%%
%% This work is distributed under the LaTeX Project Public License (LPPL)
%% ( http://www.latex-project.org/ ) version 1.3, and may be freely used,
%% distributed and modified. A copy of the LPPL, version 1.3, is included
%% in the base LaTeX documentation of all distributions of LaTeX released
%% 2003/12/01 or later.
%% Retain all contribution notices and credits.
%% ** Modified files should be clearly indicated as such, including  **
%% ** renaming them and changing author support contact information. **
%%*************************************************************************


% *** Authors should verify (and, if needed, correct) their LaTeX system  ***
% *** with the testflow diagnostic prior to trusting their LaTeX platform ***
% *** with production work. The IEEE's font choices and paper sizes can   ***
% *** trigger bugs that do not appear when using other class files.       ***                          ***
% The testflow support page is at:
% http://www.michaelshell.org/tex/testflow/



\documentclass[conference,compsoc]{IEEEtran}


% correct bad hyphenation here
\hyphenation{op-tical net-works semi-conduc-tor}
\ifCLASSINFOpdf
    \usepackage[pdftex]{graphicx}
\else
    \usepackage[dvips]{graphicx}
\fi

\usepackage{url}
\usepackage{biblatex}
\usepackage{wrapfig}
\usepackage{float}
\usepackage[hidelinks]{hyperref}
\usepackage{minted}



\graphicspath{{Abbildungen/}}
\addbibresource{literatur.bib}

\begin{document}
%
% paper title
% Titles are generally capitalized except for words such as a, an, and, as,
% at, but, by, for, in, nor, of, on, or, the, to and up, which are usually
% not capitalized unless they are the first or last word of the title.
% Linebreaks \\ can be used within to get better formatting as desired.
% Do not put math or special symbols in the title.
\title{Rescue Robot\\ \textit{Projekt angewandte Elektrotechnik}}


% author names and affiliations
% use a multiple column layout for up to three different
% affiliations
\author{\IEEEauthorblockN{Bruno Berger}
\IEEEauthorblockA{Hochschule Hamm-Lippstadt\\Interaktionstechnik und Design\\
Dr.-Arnold-Hueck-Straße 3\\ 59557 Lippstadt\\
Email: bruno.berger@stud.hshl.de}
\and
\IEEEauthorblockN{Lukas Walter}
\IEEEauthorblockA{Hochschule Hamm-Lippstadt\\Interaktionstechnik und Design\\
Dr.-Arnold-Hueck-Straße 3\\ 59557 Lippstadt\\
Email: lukas.walter@stud.hshl.de}
\and
\IEEEauthorblockN{Melanie Löbel}
\IEEEauthorblockA{Hochschule Hamm-Lippstadt\\Interaktionstechnik und Design\\
Dr.-Arnold-Hueck-Straße 3\\ 59557 Lippstadt\\
Email: melanie.loebel@stud.hshl.de}}


% make the title area
\maketitle

% As a general rule, do not put math, special symbols or citations
% in the abstract
\begin{abstract}
In Fällen wie Naturkatastrophen oder schweren Industrieunfällen ist es meist dringend notwendig, verletzte Personen zu retten oder Gegenstände zu bergen. Um dabei nicht andere Menschen den Risiken oder Gefahren auszusetzen, ist es sehr hilfreich, Roboter einzusetzen. In diesem Projekt soll ein "Rescue-Robot" entwickelt werden, der bei einem Industrieunfall wie einer Explosion seinen Einsatz findet. Bei der Explosion wurden auch radioaktive Objekte auf dem Industriegelände verteilt. Der Rescue-Robot soll sich mittels gesendeter Radiosignale autonom auf dem Industriegelände bewegen und beim Erkennen radioaktiver Objekte, diese bergen. Beim Erkennen einer Person soll über den Rescue-Robot eine Kommunikation nach außen möglich sein.
\end{abstract}

% no keywords


% For peer review papers, you can put extra information on the cover
% page as needed:
% \ifCLASSOPTIONpeerreview
% \begin{center} \bfseries EDICS Category: 3-BBND \end{center}
% \fi
%
% For peerreview papers, this IEEEtran command inserts a page break and
% creates the second title. It will be ignored for other modes.
\IEEEpeerreviewmaketitle



\section{Einführung}
Die Domäne Rettungsrobotik verfügt über sehr vielversprechende wirtschaftliche Aspekte. Menschliche Rettungskräfte sind bei Katastrophenszenarien eine knappe Ressource. Ein einzelner Bediener sollte daher idealerweise eine Vielzahl von Robotern überwachen. Beim Einsatz solcher Rettungsroboter ist ein hohes Maß an Autonomie gefordert. Sie begegnen einer Vielzahl unbekannter Objekte. Sie sollten in der Lage sein, Objekte oder ihre Umgebung zu erkennen und spezifiziert darauf zu reagieren. Es gibt zwei Entwicklungsziele, die schwer zu kombinieren sind. Zum einen ist für die Umsetzung eines autonomen Systems High-Technology notwendig. Zum anderen gibt es den Bedarf möglichst einfache und kostengünstige Systeme zu entwickeln \cite{birk2006rescue}. Bei diesem Projekt des "Rescue Robots“ liegt der Hauptfokus zunächst auf der Umsetzung der Funktionalität.

\hfill mds

\hfill August 27, 2020

\subsection{Motivation}
In Fällen wie Naturkatastrophen oder schweren Industrieunfällen ist es meist dringend notwendig, verletzte Personen zu retten oder Gegenstände zu bergen. Um dabei nicht andere Menschen den Risiken oder Gefahren auszusetzen, ist es sehr hilfreich, Roboter einzusetzen. In diesem Projekt soll ein "Rescue-Robot" entwickelt werden, der bei einem Industrieunfall wie einer Explosion seinen Einsatz findet. Bei der Explosion wurden auch radioaktive Objekte auf dem Industriegelände verteilt. Der Rescue-Robot soll sich mittels gesendeter Radiosignale autonom auf dem Industriegelände bewegen und beim Erkennen radioaktiver Objekte, diese bergen. Beim Erkennen einer Person soll über den Rescue-Robot eine Kommunikation nach außen möglich sein.

\section{Sketch of approach\hfill\textnormal{\emph{Löbel}}}
Nach Festlegung der funktionalen Anforderungen wurde ein erster Prototyp im 3D erstellt. Grundlage hierfür waren einzelene Paper Prototypes der Gruppenmitglieder. Nach Bewertung der einzelnen Prototypes wurden Teile bzw. Komponenten übernommen oder ergänzt. 


\subsection{3D Prototype}
Der resultierende Prototype, siehe Abb.\ref{fig:model_proto}, beinhaltete wie in den Anforderungen bereits erwähnt, einen Kettenantrieb für die Fortbewegung an Land. Für die Fortbewegung im Wasser wurden Turbine und Ruder zentral im unteren Bereich des Roboters in einem Durchgangsloch platziert. Für das Bergen von Gegenständen wurde ein schwenkbarer Greifarm mit Greifer zentral an der Fahrzeug Vorderseite positioniert. Der Greifer wurde von der GrabCAD Library importiert, \textit{\url{https://grabcad.com/library/4-bar-linkage-gripper-with-dynamixel-rx-64-1}}. Zur zusätzlichen Unterstützung beim Greifprozess wurden im Prototype zwei heraus fahrbare Stützen an Fahrzeug Vorderseite angebracht. Zum Sammeln der radioaktiven Gegenstände befindet sich im hinteren Teil des Roboters eine verschließbare Box. Daneben wurde eine offene Box eingebracht zum Laden von beispielsweise "Erste-Hilfe" Material oder einem Koffer. Vorne links am Fahrzeug wurden LIDAR Sensor inklusive Peripherie wie Kamera, Lautsprecher und Mikrophon auf einem Drehpodest platziert. Zusätzlich ist dieser Teil schwenkbar.

\begin{figure}[H]
  \centering{\includegraphics[width=1.0\linewidth]{Abbildungen/3D_model/prototype_front.png}}
  \caption{Kombinierter 3D Prototyp}
  \label{fig:model_proto}
\end{figure}
\section{Konzept\hfill\textnormal{\emph{Löbel}}}
Anschließend wurden über die objektorientierte Analyse die benötigten Klassen und Attribute definiert. 

\subsection{Technische Systemarchitektur}
Das technische System besteht aus folgenden Komponenten: Die Sensorik, die für das Senden und Empfangen von Signalen verantwortlich ist. Hier zählen unter anderem die Antennen zum Empfangen von Radiosignalen dazu. Ein LIDAR Sensor, über den Laserimpulse gesendet und reflektiertes gestreutes Licht zur Objekterkennung empfangen wird. Ein Wassersensor zur Messung der Kapazität, ob der Roboter sich an Land oder Wasser befindet und ein Kraft- und Druckaufnehmer zur Gewichtsmessung der zu bergenden Gegenständen gehören auch dazu. Die Signalberechnung bzw. -wertung wird durch einen Microcontroller übernommen. Je nach empfangenen Signalwert werden die jeweiligen Motoren des Roboters gesteuert. Es werden unter anderem Motoren für den Greifer und Greifarm benötigt, als auch für den Antrieb von Ketten, Turbine, Ruder, das Drehen der Peripherie oder auch für die ausfahrbaren Stützen und den verschließbaren Deckel der Bergungsbox. Hinzu kommt die Peripheriesteuerung für die Kommunikation, die sich aus den Objekten Kamera, Lautsprecher und Mikrophon zusammensetzt. 

\begin{figure}[H]
  \centering\includegraphics[width=1\linewidth]{Class Diagram Rescue-Robot.png}
  \caption{Class Diagram "Rescue Robot"}
  \label{ClassDiagram}
\end{figure}

\subsection{Subsysteme}
Nach Aufteilung der Objekte in die Kategorien "Signal transmit", "Signal reception“, "Signal calculation“, "Motor control“ und "Peripheral device control“ wurde eine Swimlane Analyse durchgeführt, in der der gesamte Prozessablauf beschrieben wurde. Daraus ergaben sich die folgenden Subsysteme: 

\begin{itemize}
\item das Firmengelände (siehe Abb. \ref{Systemumgebung})
\item die Signalverfolgung
\item die Fortbewegung (an Land und im Wasser)
\item die Objekterkennung (Hindernis, Person oder radioaktives Objekt)
\item die Navigation (Umfahren von Hindernissen)
\item die Kommunikation (zu Personen)
\item die Objektbergung (radioaktive Gegenstände)
\end{itemize}

Zu jedem dieser Subsysteme wurden detaillierte Aktivitätsdiagramme erstellt. Diese sind unter \textit{\url{https://github.com/BrunoBerger/Rescue-Robot.git}} im Ordner "Diagramme/Subsysteme/..." zu finden.\\ 
\\
Als erstes muss das \textit{Firmengelände} definiert sein. Die \textit{Fortbewegung} wird über die Ansteuerung der Motoren für Kettenantrieb (an Land) und Turbine und Ruder (im Wasser) realisiert. Hierzu soll über den Wassersensor die Kapazität gemessen und bewertet werden. Gleichzeitig können über die Antennen am Rescue Robot Radiosignale empfangen werden. Die Radiosignale enthalten eine ID und auch einen Standort (x- und y-Koordinate) zum Berechnen der Distanz. In Abb.\ref{Track_Signal} wird die \textit{Signalverfolgung} beschrieben. Zur Signalverfolgung wird die Antenne genutzt, welche die kürzere Distanz zum Radiosignal bzw Funkturm hat. Erst wenn die Radiosignal Distanz $<1$ ist, ist der jeweilige Funkturm erreicht.

\begin{figure}[H]
  \centering\includegraphics[width=1\linewidth]{Track_Signal.png}
  \caption{Subsystem: Aktivitätsdiagramm "Track Signal"}
  \label{Track_Signal}
\end{figure}

Zur \textit{Objekterkennung} soll der Rescue Robot nicht nur den LIDAR-Sensor nutzen, sondern parallel sollen zur Person-Erkennung auch die Kamera Bilder ausgewertet werden. Wenn eine Person erkannt wurde, soll die \textit{Kommunikation} gestartet werden. Dies passiert über Einschalten von Lautsprecher und Mikrophon. Der Rettungsdienst kann so mit der verletzten Person kommunizieren und über außen ggf. kleine "Erste Hilfe" leisten. Bei der \textit{Objektbergung} kommt es zu mehreren Schritten, siehe Abb.\ref{Rescue_object}. Wird ein Objekt erkannt, wird der Greifarm in die berechnete Position gefahren. Der Greifer öffnet sich und mittels Geiger Sensor wird die Radioaktivität des Objekts gemessen. Soll das Objekt aufgrund seines Strahlungswerts eingesammelt werden, schließt der Greifer sich. Gleichzeitig sind die Stützen des Rescue Robots ausgefahren. Nun wird der Greifarm angehoben und das Objektgewicht wird mittels Kraft- und Druckaufnehmer gemessen. Wird das maximale zulässige  Greif-Gewicht überschritten, öffnet der Greifer wieder und lässt das Objekt fallen. Ist das Objektgewicht im zulässigen Bereich, fährt der Greifarm die Position über der Box an, der Deckel wird geöffnet und das Objekt wird in die Box gelegt. Anschließend wird das aktuelle Transportgewicht berechnet. Ist dies größer als das maximal zulässige Transportgewicht, soll der Rescue Robot wieder zurück zu seinem Startpunkt auf dem Firmengelände fahren.

\begin{figure}[H]
  \centering\includegraphics[width=1\linewidth]{Objekt bergen.png}
  \caption{Subsystem: Aktivitätsdiagramm "Rescue Object"}
  \label{Rescue_object}
\end{figure}
\section{Evaluation}
In diesem Abschnitt werden die verschiedenen Implementationen vorgestellt,
die durchgeführt wurden um jeweils verschiedene Anforderungen
aus Abschnitt \ref{sec:reqs} zu erfüllen.
\begin{flushright}
\emph{Berger}
\end{flushright}

\subsection{3D Modell\hfill\textnormal{\emph{Berger}}}

Der grundsätzliche Aufbau des Roboters hat sich nach dem kombinierten Prototypen nicht stark verändert.
Die Aufgabe des finalen Modells war es also 
genauer zu definieren aus welchen Bauteilen der Roboter besteht
und zu beweisen das alle angedachten Funktionen im Gerät unterzubringen sind.

Wie in Abbildung \ref{fig:model_open_front} zu sehen, 
besteht der finale Roboter aus einer großen Grundplatte, 
die durch eine Rückwand und einen Deckel abgedeckt wird.
Dies vereinfacht die Wartung der inneren Komponenten
und erlaubt das komplette ausbauen der beiden Container.(NFR9)

\begin{figure}[H]
  \centering{\includegraphics[width=1.0\linewidth]{Abbildungen/3D_model/final_open_front_2.JPG}}
  \caption{Modell final TEST}
  \label{fig:model_open_front}
\end{figure}

Wie in Abbildung \ref{fig:model_open_front} zu erkennen
wurde der Greifarm des Roboters im Vergleich zum Prototypen deutlich verstärkt.
Dies verbessert die Fähigkeit des Arms besonders schwere Objekte zu bergen.(FR3)
Als Material für den Greifarm wurde vorerst Carbonfasern gewählt,
gut zu sehen in Abbildung \ref{fig:model_closeup}.
Dies wurde gemacht um die Hebelwirkung des Arms zu verringern,
wenn dieser sich auf weite Distanzen ausstreckt.
Ob die Konstruktion aus dem Material stark genug ist 
um auch schwere Objekte zu bergen,
muss noch getestet werden und könnte die Wahl des Materials noch verändern.

Die Ketten wurden um relativ sanfte Zähne erweitert 
um die Griff-Fähigkeit auf losem Untergrund zu verbessern.(FR1, NFR1)
Sie werden angetrieben durch jeweils einen Elektromotor,
der am hinteren Rad angebracht ist,
besonders gut zu sehen in Abbildung \ref{fig:model_open_back}.

In der Mitte der Grundplatte befindet sich eine Aussparung in der Wasser-Röhre.
Dort wird der Propeller an einer Klappe befestigt,
die wie in Abbildung \ref{fig:model_open_front} gezeigt
aufgeklappt werden kann, 
zur Installation oder Wartung.(NFR9)
Am Ende der Röhre befindet sich das Ruder, 
wie in Abbildung \ref{fig:model_open_back} zu sehen.(FR2, NFR2)

\begin{figure}[H]
  \centering{\includegraphics[width=1.0\linewidth]{Abbildungen/3D_model/final_open_back.JPG}}
  \caption{Modell back}
  \label{fig:model_open_back}
\end{figure}

Die beiden Container werden mit einfach von oben in den Roboter eingesetzt.
Sie werden festgehalten von dem Deckel und den Seitenwänden,
sowie von kleinen Erhebungen in der Grundplatte,
die die Füße der Container umschließen.
Die Klappe die den rechten Container verschließt ist in dem Deckel integriert.(FR4)

Die Stützen an der Vorderseite blieben im Ansatz gleich,
wurden aber durch eine Gummi-Dichtung in der Mitte des Beins erweitert,
zu sehen unten rechts in Abbildung \ref{fig:model_open_front}.
Sobald der Roboter Wasser erkennt, 
können die Stützen komplett eingefahren werden,
wodurch die Dichtungen das sonst leicht durchlässige Loch im Chassis abdichten.

\begin{figure}[H]
  \centering{\includegraphics[width=1.0\linewidth]{Abbildungen/3D_model/Final_utility_closeup.JPG}}
  \caption{Werkzeuge des Roboters}
  \label{fig:model_closeup}
\end{figure}

Abbildung \ref{fig:model_closeup} zeigt deutlich den Aufbau von dem Sensor-Array und dem Greifer.
Der Lautsprecher und die Kamera, Mikrofon und LIDAR Sensoren 
befinden sich auf einer kipp- und drehbaren Halterung.
Dies ermöglicht die komplette Aufnahme der Umgebung.(NFR8)

Der Greifer ist angetrieben von 2 Motoren und setzt deren Drehbewegung 
in eine parallele Bewegung der Greifer-Platten in rot um.
Dies ermöglicht sanftes und rutschfestes Greifen von viele unterschiedlichen Formen.(NFR3.1)

Rechts im Hintergrund von Abbildung \ref{fig:model_closeup} befindet sich der offene Container.
Zu erkennen sind die vier Haken, 
an denen eventuelles Erste-Hilfe-Material oder ähnliches auf gehangen werden kann,
um den Container aufgeräumt zu halten.(NFR5)



\subsection{Implementierung in C\#\hfill\textnormal{\emph{Walter}}}
Die Software wurde mithilfe von C\# implementiert, dies hat den Hintergrund, dass Unity diese Sprache verwendet. So konnte das ganze Projekt in einer Sprache einheitlich implementiert werden. 

C\# eignet sich durch seine Objekt orientiertheit sehr gut für die Beschreibung Komplexer Systeme. Jedes Objekt welches in dem behandelten Szenario vorhanden ist, wurde im Code implementiert. Die vorhandenen Klassen werden in einem Klassen-Diagramm in Abbildung \ref{ClassDiagram} illustriert.

Jede dieser Klassen hat verschiedene Methoden und Attribute. Die Objekte werden in Abschnitt \ref{obj} weiter erläutert. 

\subsubsection{Objekte}
\label{obj}
Die Simulation des Roboters sollte so genau wie möglich sein, daher wurden alle Gegenstände die mit dem Roboter Interagieren eigenständige Objekte. Bei den Objekten kann es sich um sogenannte Ground-Objekte handeln, welche jedes Objekt beschreiben, welches auf der Karte auftauchen kann. So erben Die Obstacle-Objekte von den Ground-Objekten die Position und das Traversable Attribut. Dies wird in Abbildung \ref{erben} illustriert. 

\begin{figure}[H]
  \centering{\includegraphics[width=1.0\linewidth]{Abbildungen/implementierung/vererbung.PNG}}
  \caption{Vererbung von Ground-Objekt zu Obstacle-Objekt}
  \label{erben}
\end{figure}
Zusätzlich zu den Ground-Objekten gibt es Peripherie Geräte für den Roboter, wie zum Beispiel die Kamera oder das Mikrophon. Auch Bauteile wie Motoren wurde mit verschiedenen Eigenschaften implementiert. So kann ein Motor zum Beispiel einen Namen, eine Anfangs- und End-position, eine Geschwindigkeit und einen State besitzen.

Die Rescue Bot Klasse, welche in Abbildung \ref{bot2} zu sehen ist, beinhaltet alle Sensoren und Aktoren, welche durch den Roboter miteinander interagieren. 

\begin{figure}[H]
  \centering{\includegraphics[width=1.0\linewidth]{Abbildungen/implementierung/rescueBotClass2.PNG}}
  \caption{Rescue Bot Klasse}
  \label{bot2}
\end{figure}
So befindet sich zum Beispiel der LIDAR Sensor, der Grappler und der Geigerzähler innerhalb der Rescue Bot Klasse. Durch diese Verkettung kann der Rescue Bot auf die Methoden der Geräte welche auch wieder Klassen sind zugreifen. 


\subsubsection{Karte}
\label{map}
Die Karte wird aus einem zweidimensionalen Array generiert. Das Eingabe Array besteht dabei aus verschieden Strings. Aus diesem Input wird mithilfe einer Switch-Case Anweisung ein neues Array generiert, welches aus Objekten besteht, die den Strings innerhalb des Eingabe Arrays entsprechen. Das Eingabe Array wird in Abbildung \ref{map} dargestellt. 

\begin{figure}[H]
  \centering{\includegraphics[width=1.0\linewidth]{Abbildungen/implementierung/mapArr.PNG}}
  \caption{Zweidiemensionales Array als Karte}
  \label{map}
\end{figure}

So wird zum Beispiel aus einem "R“ innerhalb des Eingabe Arrays ein Radioaktives Objekt erstellt, welches eine zufällige Größe, Gewicht und Strahlung als Attribute besitzt. Der Prozess der Objekt Instanziierung kann in Abbildung \ref{rad} betrachtet werden. 

Eine Aufschlüsselung der einzelnen Buchstaben und ihrer Bedeutung wird in Folgender Tabelle erläutert:
\begin{table}[H]
\centering
\begin{tabular}{l|l}
Zeichen & Bedeutung                 \\ 
\hline
0       & Frei befahrbarer Untergrund             \\
W       & Wasser                    \\
X       & Wand / Äußere Begrenzung  \\
S       & Startpunkt                \\
F       & Funkturm                  \\
R       & Radioaktives Objekt       \\
P       & Person  
\label{buchstaben}
\end{tabular}
\end{table}

Die Objekte werden direkt in das objArr Array gespeichert, was beispielhaft in der vorletzten Zeile in Abbildung \ref{map} zu sehen ist. 


\begin{figure}[H]
  \centering{\includegraphics[width=1.0\linewidth]{Abbildungen/implementierung/createRadObj.PNG}}
  \caption{Instanziierung von Radioaktiven Objekten}
  \label{rad}
\end{figure}

\subsubsection{Erkennung und Bergen von Radioaktiven Gegenständen}
\label{erk}
Die Erkennung und das Bergen von Radioaktiven Gegenständen ist eins der Hauptfeatures des Rescue Bots. Die Erkennung der Objekte erfolgt in mehreren schritten. Bevor der Bot fährt wird über den LIDAR Sensor die unmittelbare Umgebung gescannt. Hierbei werden die Punkte vor, hinter, links, rechts und unter dem Bot erkannt. Felder welche Diagonal zum Bot liegen werden ignoriert. Sollte sich auf einem Feld welches direkt an den Bot grenzt ein Radioaktives Objekt liegen, wird dieses aufgesammelt wenn es bestimmten Kriterien entspricht. Die Kriterien sind, die Größe, die Abgegebene Radioaktive Strahlung und das Gewicht. 

Zu erst wird die Größe des Objektes mithilfe das LIDAR Sensors geprüft. Liegt das Objekt unterhalb der Maximalen Größe wird der Greifer in Richtung des Objektes bewegt und die Radioaktive Strahlung durch den Geigerzähler gemessen. Sollte die Strahlung einen Bestimmten Wert übersteigen wir das Objekt gegriffen und das Gewicht gemessen. Ist auch das Gewicht geringer als das maximale Gewicht, kann der Gegenstand bewegt eingesammelt werden. 

Dieser Prozess kann auch anhand des Aktivitätsdiagrams in Abbildung \ref{Rescue_object} nachvollzogen werden.
\subsubsection{Navigation}
\label{nav}
Der Bot soll Autonom durch eine generierte Karte fahren. Zur Hilfestellung gibt es innerhalb der Karte drei Funktürme. 

Jeder dieser Funktürme hat eine ID und eine Position auf der Karte. Diese Daten kann jeder der Funktürme an den Rescue Bot senden. Aus diesen Daten kann der Bot entscheiden welcher Funkturm der nächste ist und fährt in diese Richtung. 

Die Berechnung der Distanz wird in Abbildung \ref{dist} gezeigt. 
\begin{figure}[H]
  \centering{\includegraphics[width=1.0\linewidth]{Abbildungen/implementierung/calcDist.PNG}}
  \caption{Berechnung der Distanz zwischen zwei Punkten auf einer Fläche durch den Satz des Pythagoras}
  \label{dist}
\end{figure}

Entspricht die gemessene Distanz einer Längeneinheit (dist = 1), steht der Bot direkt neben dem Funkturm. Um den Funkturm für weitere Berechnungen zu ignorieren, wird die Distanz auf eine hohe Zahl gesetzt, im Beispiel in Abbildung \ref{dist} ist dies 100. Dies hat zur Folge, dass nun die Kürzeste Distanz zwischen den zwei verbleibenden Funktürmen gewählt wird.

Um sich zu den verschiedenen Funktürmen zu bewegen, können verschiedene Ansätze gewählt werden. 

In diesem Fall wird die Differenz der X- und Y-Koordinate berechnet und anschließend versucht die Differenz auf Null zu bringen. Dies geschieht indem zwischen einer Bewegung in X-Richtung und in Y-Richtung gewechselt wird, um eine Diagonale Bewegung zu ermöglichen. Auf diese Art und Weise kann die gefahrene Strecke minimiert werden. 

Sollten sich Hindernisse auf dem Weg befinden, versucht der Bot in eine zufällige Richtung auszuweichen und versucht es erneut, solange bis das Hindernis umfahren ist.

\subsubsection{Test Case}
\label{test}
Zum Testen des Systems wurde ein Szenario entwickelt, welches einige der Funktionen des Roboters überprüft. 

So soll getestet werden ob der Roboter den Rotor einschaltet, wenn er schwimmt, ob er Radioaktive Gegenstände eigenständig findet und einsammelt und ob einer eine Person erkennt und mit ihr interagiert. 

Folgende Einschränkungen wurden dabei festgelegt: 

\begin{itemize}
	\item Es gibt genau eine Person, welche sich an einem der in Abschnitt \ref{nav} beschriebenen Funktürme befinden muss
	\item Die Radioaktiven Gegenstände dürfen frei verteilt werden
	\item Hindernisse dürfen frei verteilt werden
	\item Es wird auf Sicht Einschränkungen durch Nebel verzichtet
	\item Bewegliche Hindernisse werden nicht berücksichtigt und entfernt
\end{itemize}
 
Das Ziel des Test Case ist, dass der Roboter So lange an Funktürmen nach einer Person sucht bis er sie gefunden hat. Während er auf der Suche ist, soll er jedes Radioaktive Objekt einsammeln, welches er findet, solange er nicht voll beladen ist. Außerdem soll angezeigt werden, welcher Motor aktuell verwendet wird.

Der Programmablauf ist als erfolgreich anzusehen, wenn die Person gefunden wurde und die Aktuelle Ladung ausgegeben wurde.

\begin{figure}[H]
  \centering{\includegraphics[width=1.0\linewidth]{Abbildungen/implementierung/output.PNG}}
  \caption{Ausgabe des Programms (Einsammeln von Radioaktiven Gegenständen und Interaktion mit Personen)}
  \label{output}
\end{figure}

\begin{figure}[H]
  \centering{\includegraphics[width=1.0\linewidth]{Abbildungen/implementierung/waterr.PNG}}
  \caption{Ausgabe des Programms (Verwendung des Aktuellen Motors)}
  \label{waterr}
\end{figure}

Wie in Abbildung \ref{output} abgelesen werden kann, findet der Rescue Bot ein Radioaktives Objekt, Fährt den Greifer aus und entfernt es. Im weiteren Programmverlauf findet der Rescue Bot die Person und hat begonnen mit ihr zu interagieren. Somit ist das Einsammeln von Radioaktiven Gegenständen und die Interaktion mit Personen funktionsfähig. 

In Abbildung \ref{waterr} wird gezeigt welche Motoren aktuell verwendet werden und welche abgeschaltet sind. Fährt der Rescue Bot auf der Wasser Oberfläche so ist die Turbine aktiv, fährt er auf Land sind die Beiden Motoren für den Kettenantrieb aktiviert. Somit konnte auch diese Funktion validiert werden.

\subsection{Implementierung in Unity\hfill\textnormal{\emph{Berger}}}
Zusätzlich wurde der wurde Wechsel der Antriebe des Roboters 
noch einmal genauer in Unity implementiert.
Hierfür wurde das Modell aus Solidworks importiert 
und eine Beispiel Landschaft mit einem Wasser-Teich erstellt.

\begin{figure}[H]
  \centering{\includegraphics[width=0.8\linewidth]{Abbildungen/unity1.png}}
  \caption{Robot in Unity}
  \label{fig:unity1}
\end{figure}


\subsection{Simulation des Greifarms\hfill\textnormal{\emph{Berger}}}
writen in python 

export with \footnote{http://wiki.ros.org/sw\_urdf\_exporter}

Das Skript generiert aus den URDF-Daten eine sogenannte "Chain".
Eine zweite Funktion nimmt diese Chain und ein Array mit Koordinaten entgegen
und gibt die Ausrichtung der Kettenglieder aus.
Diese ausgerichtete Kette wird dann, 
wie in Abbildung \ref{fig:greifarm1} dargestellt,
mit dem Matplotlib-Package visualisiert werden.

\begin{figure}[H]
  \centering{\includegraphics[width=1.0\linewidth]{Abbildungen/greifarm1.png}}
  \caption{Matplotlib ouput}
  \label{fig:greifarm1}
\end{figure}

Alternativ könnten diese Daten auch an einen tatsächlichen Roboter-Arm weitergegeben werden.
Der Rescue-Robot müsste nur die relative Position des zu greifenden Objekts berechnen
und könnte dann das Objekt greifen und dann eine Position über dem Container anfahren.

\section{Zusammenfassung und Ausblick\hfill\textnormal{\emph{Berger}}}
Zusammenfassend lässt sich sagen das alle Anforderungen aus Abschnitt \ref{sec:reqs} erfüllt wurden.
Eine genauere Darstellung von welchen Implementierungen die einzelnen Anforderungen erfüllt werden,
lässt sich in der finalen Präsentation im GitHub Repository finden.
\footnote{https://github.com/BrunoBerger/Rescue-Robot/\linebreak
    blob/master/Präsentation/Final\_Presentation/\linebreak
    Final\_Project\_RescueRobot.pptx}
\\  

Als Weiterführung könnten weitere Tests mit dem 3D Modell durchgeführt werden,
um die Schwimmfähigkeit und die Effektivität des Propellers zu beweisen.
Erste flow-simulation in SolidWorks haben bereits einen recht schnellen Fluss
durch die Röhre des Roboters gezeigt.
Zusätzlich könnte die Stabilität des Greifers unter Last geprüft werden.

Auch das Design könnte in Zukunft noch verfeinert werden,
mit detaillierteren Hardware Schnittstellen 
und Protokollen für Randsituationen.

Die Software könnte noch um ein paar Features erweitert werden,
wie zum Beispiel diagonales Fahren des Roboters 
oder Sichteinschränkungen wie Nebel.

    


\section{Anhang}

\subsection{GitHub Übersicht\hfill\textnormal{\emph{Berger}}}

Zu Versionskontrolle wurde ein Git Repository genutzt 
das auf GitHub gehostet wird.
Darüber konnten wir gleichzeitig gemeinsam an dem Projekt arbeiten,
was auch in sehr gleichen Anteilen stattgefunden hat.
Die gesamte Arbeitszeit würden wir auf ungefähr 500 Stunden schätzen.

Zusätzlich wurden weitere Features von Github genutzt, 
wie die Project-Boards,
die als Erweiterung des Trello-Boards genutzt wurden 
um die Aufgaben für die jeweiligen Meilensteine festzuhalten.

Auch wurde GitHub-Actions genutzt 
um "Nightly Builds" von der Implementierung in C\# zu generieren.
Dabei wird das Projekt auf einer virtuellen Windows Installation gebaut 
und dann der Status des Builds dann in der Readme der Repositorys angezeigt.
Dies sieht aus wie in Abbildung \ref{fig:buildBadge}, 
falls der Build Vorgang nicht fehlschlägt
und Informiert schnell über den aktuellen Stand des Projekts.

\begin{figure}[H]
  \centering{\includegraphics[width=0.3\linewidth]{Abbildungen/buildBadge.png}}
  \caption{Build Badge}
  \label{fig:buildBadge}
\end{figure}


\subsection{Source Code}

Der Quellcode für alle Implementierungen ist auf GitHub zu finden.
\footnote{https://github.com/BrunoBerger/Rescue-Robot/tree/master/Implementierung} % conference papers do not normally have an appendix

\section{Affidavit}
We (Bruno Berger, Lukas Walter, Melanie Löbel) herewith declare that we have composed the present paper and work ourself and without use of any other than the cited sources and aids. Sentences or parts of sentences quoted literally are marked as such; other references with regard to the statement and scope are indicated by full details of the publications concerned. The paper and work in the same or similar form has not been submitted to any examination body and has not been published. This paper was not yet, even in part, used in another examination or as a course performance.


% use section* for acknowledgment
\ifCLASSOPTIONcompsoc
  % The Computer Society usually uses the plural form
  \section*{Acknowledgments}
\else
  % regular IEEE prefers the singular form
  \section*{Acknowledgment}
\fi


The authors would like to thank...





% trigger a \newpage just before the given reference
% number - used to balance the columns on the last page
% adjust value as needed - may need to be readjusted if
% the document is modified later
%\IEEEtriggeratref{8}
% The "triggered" command can be changed if desired:
%\IEEEtriggercmd{\enlargethispage{-5in}}

% references section

\printbibliography


% that's all folks
\end{document}
