
\section{Einführung}
Die Domäne Rettungsrobotik verfügt über sehr vielversprechende wirtschaftliche Aspekte. Menschliche Rettungskräfte sind bei Katastrophenszenarien eine knappe Ressource. Ein einzelner Bediener sollte daher idealerweise eine Vielzahl von Robotern überwachen. Beim Einsatz solcher Rettungsroboter ist ein hohes Maß an Autonomie gefordert. Sie begegnen einer Vielzahl unbekannter Objekte. Sie sollten in der Lage sein, Objekte oder ihre Umgebung zu erkennen und spezifiziert darauf zu reagieren. Es gibt zwei Entwicklungsziele, die schwer zu kombinieren sind. Zum einen ist für die Umsetzung eines autonomen Systems High-Technology notwendig. Zum anderen gibt es den Bedarf möglichst einfache und kostengünstige Systeme zu entwickeln \cite{birk2006rescue}. Bei diesem Projekt des "Rescue Robots“ liegt der Hauptfokus zunächst auf der Umsetzung der Funktionalität.

\hfill mds

\hfill August 27, 2020

\subsection{Motivation}
In Fällen wie Naturkatastrophen oder schweren Industrieunfällen ist es meist dringend notwendig, verletzte Personen zu retten oder Gegenstände zu bergen. Um dabei nicht andere Menschen den Risiken oder Gefahren auszusetzen, ist es sehr hilfreich, Roboter einzusetzen. In diesem Projekt soll ein "Rescue-Robot" entwickelt werden, der bei einem Industrieunfall wie einer Explosion seinen Einsatz findet. Bei der Explosion wurden auch radioaktive Objekte auf dem Industriegelände verteilt. Der Rescue-Robot soll sich mittels gesendeter Radiosignale autonom auf dem Industriegelände bewegen und beim Erkennen radioaktiver Objekte, diese bergen. Beim Erkennen einer Person soll über den Rescue-Robot eine Kommunikation nach außen möglich sein.
